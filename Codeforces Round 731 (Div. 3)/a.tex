\documentclass{beamer}
\usetheme{JuanLesPins}
\usepackage[sort]{natbib}
\usepackage{parskip}
\usepackage[utf8x]{inputenc}
\usepackage{graphicx}
\usepackage{booktabs}
\usepackage{url}
\usepackage{biblatex}
\addbibresource{ref.bib}
\usepackage{xcolor,colortbl} %package for table bg
\setlength{\parskip}{\smallskipamount} 

\title[Group 8]{Insertion and Radix Sort}
\subtitle{}
\institute[\LaTeX-LT]{Group 8}
\author{Jeasmin(2003017)\\Jerin(2003032)\\Neha(2003034)\\Srishty(2003042)\\Ania(2003048)\\Sara(2003058)}



\begin{document}
  \frame[plain]{\titlepage}
%table of conents
 \frame[t]{ \frametitle{Presentation outline} \tableofcontents }

	\section[Insertion Sort]{Insertion Sort}
	\subsection[Definition]{Definition}
%frame 1 definition and example
    \frame[t]
    {\frametitle{Insertion Sort}   \vspace{7mm}
    \textcolor{red}{\textbf{Insertion sort}}\cite{isDF}  is a sorting algorithm that places an unsorted element at
     its suitable place in each iteration.\\
    \vspace{5mm}
    \textbf{Example:}\\
%example images
    \begin{columns}
        \column{0.4\textwidth}
        \includegraphics[width=\textwidth]{exmpl.jpg}
    \end{columns}
  \centering{ Fig 1\cite{isDF}: Sorting an array using insertion sort}
    }
    \subsection[Example]{Example of insertion sort}
    %frame 2 step by step
    \frame[t]
    {
    \frametitle{How does it work?}
    
    \begin{columns}
        \column{0.4\textwidth}
        \vspace{7mm}
         %1st image
        \includegraphics[width=\textwidth]{is1.jpg}
        \cite{isDF}
    
        \vspace{2mm}
       
        \column{0.6\textwidth}
        \begin{itemize}[<+-|alert@+>]
        \item Initial array
        \end{itemize}
    \end{columns}
    \textbf{Step 1}
    \begin{columns}
        \column{0.4\textwidth}
        \vspace{7mm}
         %2nd image
        \includegraphics[width=\textwidth]{is2.jpg}\cite{isDF}
    
        \vspace{2mm}
        
        \column{0.6\textwidth}
        \begin{itemize}[<+-|alert@+>]
      
        \item key=5
        \item Copy 9 at index 2
        \item Replace value of 1st position with key
        \end{itemize}
    \end{columns}
 
    }
%new frame
    \frame[t]
    {
    \frametitle{How does it work?}
    \textbf{Step 2}
    \begin{columns}
        \column{0.4\textwidth}
        \vspace{7mm}
         %3rd image
        \includegraphics[width=\textwidth]{is3.jpg}\cite{isDF}
    
        \vspace{2mm}
        
        \column{0.6\textwidth}
        \begin{itemize}[<+-|alert@+>]
        \item key=1
        \end{itemize}
    \end{columns}
    }
    %new frame
    \frame[t]
    {
    \frametitle{How does it work?}
    \textbf{Step 3}
    \begin{columns}
        \column{0.4\textwidth}
        \vspace{7mm}
         %4th image
        \includegraphics[width=\textwidth]{is4.jpg}\cite{isDF}
    
        \vspace{2mm}
        
        \column{0.6\textwidth}
        \begin{itemize}[<+-|alert@+>]
        \item key=4
        \end{itemize}
    \end{columns}
    }
    %new frame
    \frame[t]
    {
        \frametitle{How does it work?}
        \textbf{Step 4}
        \begin{columns}
            \column{0.4\textwidth}
            \vspace{7mm}
            %5th image
            \includegraphics[width=\textwidth]{is5.jpg}
        
            \vspace{2mm}
            
            \column{0.6\textwidth}
            \begin{itemize}[<+-|alert@+>]
            \item key=3
            \end{itemize}
        \end{columns}
        }
 %frame 3 psudocode 
\subsection[Psudocode]{Psudocode of insertion sort}
\frame[t]{
    \frametitle{Psudocode}
    \begin{columns}
        \column{0.8\textwidth}
        \vspace{3mm}
        %image
        \includegraphics[width=\textwidth]{ispsudo.jpg}
    \end{columns}
    
}
%frame 4 time complexity
\subsection[Uses]{Advantages/Disadvantages}
\frame[t]{
    \frametitle{Advantages/Disadvantages}
    %table 
    \begin{table}[!hbt]
        \centering
        \setlength{\tabcolsep}{0.2cm}
        \renewcommand{\arraystretch}{1.5}
        \label{tab-marks}
       
        \begin{tabular}{|p{5cm}|p{5cm}|}
        \hline \textbf{Advantages}\cite{ISad}  & \textbf{Disadvantages}\cite{ISad} \\
        \hline \rowcolor{lightgray} The main advantage of the insertion sort is its simplicity. & 
        The disadvantage of the insertion sort is that it does not perform as well as other, better sorting algorithms \\
        \hline It also exhibits a good performance when dealing with a small list. &
        With n-squared steps required for every n element to be sorted, the insertion sort does not deal well with a huge list.\\
        \hline \rowcolor{lightgray}The insertion sort is an in-place sorting algorithm so the space requirement is minimal. &
        The insertion sort is particularly useful only when sorting a list of few items.\\
        \hline
        \end{tabular}
        \end{table}
    
}
\section[Radix Sort]{Radix Sort}
\subsection[Definition]{Definition}
%frame 1 definition and example
\frame[t]
{\frametitle{Radix Sort}   \vspace{1mm}
\textcolor{red}{\textbf{Radix sort}}\cite{radDef}  is a non-comparative sorting algorithm. Radix sort was developed to sort large integers. As an integer is treated as a string of digits so we can also call it a string sorting algorithm.\\
\vspace{5mm}
\textbf{Example:}\\
%example images
\begin{columns}
    \column{0.6\textwidth}
    \includegraphics[width=\textwidth]{sorted.jpg}
    
\end{columns}
\centering{ Fig 1\cite{radDef}: Sorting an array using radix sort}
}

\subsection[Example]{Example of radix sort}
%frame 2 step by step
\frame[t]
{
\frametitle{How does it work?}

\begin{columns}
    \column{0.8\textwidth}
    \vspace{7mm}
     %1st image
    \includegraphics[width=\textwidth]{Radix-Sort-Array.png}\cite{rad2}

    \vspace{2mm}
   
    \column{0.6\textwidth}
    \begin{itemize}[<+-|alert@+>]
    \item Initial array
    \end{itemize}
\end{columns}
\textbf{Step 1}
}

\frame[t]
{
\frametitle{How does it work?}

\begin{columns}
    \column{0.8\textwidth}
    \vspace{7mm}
     %2nd image
    \includegraphics[width=\textwidth]{Radix-Sort--1.png}\cite{rad2}

    \vspace{2mm}
    
    \column{0.6\textwidth}
    \begin{itemize}[<+-|alert@+>]
  
    \item focus on least  \\
    significant bit(LSB) \\
    and sort array \\
    according to LSB
    \end{itemize}
    
\end{columns}
\textbf{Step 2}
}
%new frame
\frame[t]
{
\frametitle{How does it work?}
\begin{columns}
    \column{0.8\textwidth}
    \vspace{7mm}
     %3rd image
    \includegraphics[width=\textwidth]{Radix-Sort--2.png}\cite{rad2}

    \vspace{2mm}
    
    \column{0.6\textwidth}
    \begin{itemize}[<+-|alert@+>]
    \item sort according to\\
     10th possition
    \end{itemize}
    
\end{columns}
\textbf{Step 3}
}
%new frame
\frame[t]
{
\frametitle{How does it work?}

\begin{columns}
    \column{0.8\textwidth}
    \vspace{7mm}
     %4th image
    \includegraphics[width=\textwidth]{Radix-Sort--3.png}\cite{rad2}

    \vspace{2mm}
    
    \column{0.6\textwidth}
    \begin{itemize}[<+-|alert@+>]
    \item sort according to\\
     100th possition
    \end{itemize}
   
\end{columns}
\textbf{Step 4}
}
%new frame
\frame[t]
{
    \frametitle{How does it work?}

    \begin{columns}
        \column{0.8\textwidth}
        \vspace{7mm}
        %5th image
        \includegraphics[width=\textwidth]{Radix-Sort--4.png}\cite{rad2}
    
        \vspace{2mm}
        
        \column{0.6\textwidth}
        \begin{itemize}[<+-|alert@+>]
        \item sorted array
        \end{itemize}
        
    \end{columns}
    }
    
%frame 3 psudocode 
\subsection[Psudocode]{Psudocode of radix sort}
\frame[t]{
\frametitle{Psudocode}
\begin{columns}
    \column{0.8\textwidth}
    \vspace{3mm}
    %image
    \includegraphics[width=\textwidth]{pseudocode.png}\cite{radDef}
\end{columns}

}
%frame 4 time complexity
\subsection[Uses]{Advantages/Disadvantages}
\frame[t]{
    \frametitle{Advantages/Disadvantages}
    %table 
    \begin{table}[!hbt]
        \centering
        \setlength{\tabcolsep}{0.3cm}
        \renewcommand{\arraystretch}{1.5}
        \label{tab-marks1}
       
        \begin{tabular}{|p{5cm}|p{5cm}|}
        \hline \textbf{Advantages}\cite{rad3}  & \textbf{Disadvantages}\cite{rad3} \\
        \hline \rowcolor{lightgray} Radix sort is fast when the keys are short i.e. when the range of the array elements is less. & 
        Since it depends on digits or letters, Radix Sort is much less flexible than other sorts.\\
        \hline It is a stable sorting algorithm, meaning that elements with same key value maintain their relative order in sorted output. &
        It requires a significant amount of memory to hold the count of the number of times each digit value appears.\\
        \hline \rowcolor{lightgray}It has a linear time complexity,which makes it faster than comparison-based sorting algorithms &
        It is not efficient for small data sets or data sets with a small number of unique keys.\\
        \hline
        \end{tabular}
        \end{table}
    
}

%new sec
\section[Complexity]{Complexity }
\subsection{Time \& Space Complexity}
%time and space complexity frame
\frame[t]{
    \frametitle{Time \& Space Complexity}
    \textbf{\underline{Time Complexity:}}\\
    \vspace{3mm}
    \textbf{Insertion Sort} has a time complexity of $O(n^2)$ for average case and worst case
    and $O(n)$ for best case, where $n$ is the size of array.\\
    \vspace{3mm}
    \textbf{Radix Sort} has a time complexity of $O(nd)$, where $n$ is the size of array and $d$ is 
    number of digits.\\
    \vspace{5mm}
    \textbf{\underline{Space Complexity:}}\\
    \vspace{3mm}
    \textbf{Insertion Sort} has a space complexity of $O(1)$.\\
    \vspace{3mm}
    \textbf{Radix Sort} has a space complexity of $O(n+d)$. 
}
\frame[t]{
    \frametitle{References}
 
 \printbibliography
}
\frame[t]{
\frametitle{}
\vfill
\centering
\includegraphics[width=\textwidth]{thankyou.png}
}

\end{document}